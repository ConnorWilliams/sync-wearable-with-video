\chapter{Conclusion}
\label{chap:conclusion}
% A compulsory chapter, of roughly 5 pages
%
% The chapter will consist of three parts:
%     1)  (Re)summarise the main contributions and achievements, summing up the content.
%     2)  * Clearly state the current project status e.g. 'X is working, Y
%             is not'.
%         * Evaluate what has been achieved with respect to the initial aims and
%             objectives e.g. 'I completed aim X outlined previously, the evidence
%             for this is within Chapter Y'. There is no problem including aims
%             which were not completed, but it is important to evaluate and/or
%             justify why this is the case.
%     3)  * Outline any open problems or future plans.
%         * Unexplored options or interesting outcomes.
%         * What you could have done given more time.
%         * e.g.'my experiment for X gave counter-intuitive results, this could be
%             because Y and would form an interesting area for further study'
%         * e.g.'users found feature Z of my software difficult to use, which is
%             obvious in hindsight but not during at design stage; to resolve this,
%             I could clearly apply the technique of Smith'.

Future work: Use these assessments as a means of rejecting false positive skeletons.
Skeletons are the format in which the data comes from the cameras. It contains
information on the location of the main joints of the person i.e. elbow, wrist,
knee etc.
