\chapter*{Abstract}
\addcontentsline{toc}{chapter}{Abstract}
% A compulsory section, of at most 1 page
% A summary or abstract of:
%     * Project context.
%         This section is a (very) short version of what is typically the first
%         chapter.  Note that for research-type projects, this must include
%         a clear research hypothesis.  This will obviously differ significantly
%         for each project, but an example might be as follows:
%
%         My research hypothesis is that a suitable genetic algorithm will yield
%         more accurate results (when applied to the standard ACME data set) than
%         the algorithm proposed by Jones and Smith, while also executing in less
%         time.
%
%     * Aims and objectives.
%     * Main contributions, deliverables and achievements eg:
%         * I spent $120$ hours collecting material on and learning about the
%               Java garbage-collection sub-system.
%         * I wrote a total of $5000$ lines of source code, comprising a Linux
%               device driver for a robot (in C) and a GUI (in Java) that is
%               used to control it.
%         * I designed a new algorithm for computing the non-linear mapping
%               from A-space to B-space using a genetic algorithm, see page $17$.
%         * I implemented a version of the algorithm proposed by Jones and
%               Smith in [6], see page $12$, corrected a mistake in it, and
%               compared the results with several alternatives.
%
% The goal is to ensure the reader is clear about what the topic is,
% what you have done within this topic, and what your view of the outcome is.

Obesity, depression, stroke, falls, cardiovascular and musculoskeletal disease
are some of the biggest health issues and fastest-rising categories of healthcare
costs in the UK. The associated expenditure is widely regarded as unsustainable and the
impact on quality of life is felt by millions of people in the UK each day. With a
rapidly ageing population - could technology be the answer to some of these problems?

SPHERE (a Sensor Platform for HEalthcare in a Residential Environment)
is developing a number of different sensors that will combine to build a
picture of how we live in our homes. This information can then be used to spot issues
that might indicate a medical or well-being problem.

The technology could help by:
\begin{itemize}
    \item Predicting falls and detecting strokes so that help may be summoned.
    \item Analysing eating behaviour - including whether people are taking prescribed medication.
    \item Detecting periods of depression or anxiety.
\end{itemize}
\cite{sphere_website}

Within the SPHERE research group, one of the sensors being developed is a wearable accelerometer.
This accelerometer induces a drift between the true time and the local time on
the device large enough that the wearable needs to be re-synchronised for every use.

This document looks to consider video data from RGBD cameras around the house,
which provides data on where the person is, together with the acceleration data
from the wearable sensor. With this data it should be possible to design an
algorithm which automatically corrects the time drift so that the accelerometer
and the video remain synchronised.

This project has been broken down in the following steps:
\begin{enumerate}
    \item Consider different types of temporal distortions which could take place
        on a device such as the one in question.
    \item Implement a program which generates synthetic data and can apply any combination
        of the different types of temporal distortion.
    \item Devise a method to compensate for the different temporal distortions.
    \item Given a synchronised set of data from the accelerometer and the
        cameras, introduce artificial drifts between data sets and assess how well
        the method work.
    \item Apply this method to real drifted data which has already been
        collected in the SPHERE house in three locations; hallway, living room and kitchen.
\end{enumerate}

Implicit to all of these steps was a large amount of data formatting and
visualisation.
