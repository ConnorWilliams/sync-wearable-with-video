\chapter*{Notation and Acronyms}
\addcontentsline{toc}{chapter}{Notation and Acronyms}
% For an acronym, this can be introduced at the first point of use via
% 'Advanced Encryption Standard (AES)' or similar, noting the capitalisation of
% relevant letters. An example of a table is as follows:
%
% AES         Advanced Encryption Standard
% DES         Data Encryption Standard
% H(x)        Hamming weight of x
% F_q         A finite field with q elements

\begin{tabular}{lcl}
	SPHERE &:     			& a Sensor Platform for HEalth in a Residential Environment \\
	RGB-D Camera &:     	& ASUS Xtion PRO cameras which can track colour as well as depth. \\
	Accelerometer &:     	& A three axis accelerometer on the dominant wrist, attached using a strap \\
    Temporal Distortion &:  & A discrepancy in the timestamp of the data \\
    No Distortion &:     	& No discrepancy \\
    Constant Distortion &:  & Clock constantly offset from the ground truth clock \\
    Linear Distortion &:    & Clock slower or faster than the ground truth clock \\
    Periodic Distortion &:  & Clock sometimes is faster, sometimes is slower than the ground truth clock according to a sinusoidal wave \\
    Triangular Distortion &:& Clock sometimes is faster, sometimes is slower than the ground truth clock according to a triangular wave \\
\end{tabular}
