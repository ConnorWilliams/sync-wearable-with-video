\chapter{Project Execution}
\label{chap:execution}
% A topic-specific chapter of roughly 15 pages
%
% Describe what you did.
% Explain the main activity or activities during the project.
% The content is highly topic-specific.
%
% Maybe split the chapter into two sections:
%     * One will discuss the design of something (e.g., some hardware or software, or
%         an algorithm, or experiment), including any rationale or decisions made.
%     * The other will discuss how this design was realised via some form of
%         implementation.
%
% It is common to include evidence of best practice project management:
%     * Use of version control.
%     * Choice of programming language.
%
% Rather than simply a list, make sure any such content is informative in some
% way: for example, if there was a decision to be made then explain the trade-offs
% and implications involved.

\section{Data Generator}
This section is about my data generator:
\begin{itemize}
    \item Why is it necessary?
    \item What features does it have?
    \item Why does it have these features? (Linked to different temporal distortions)
\end{itemize}

\section{Methods}
This section will be about the methods I have found which solve the different
types of temporal distortion.
\begin{itemize}
    \item What are the methods?
    \item How well do the methods work?
    \item Do other temporal distortions affect the performance?
    \item What combination of distortions are amendanble?
\end{itemize}
