\chapter{Critical Evaluation}
\label{chap:evaluation}
% A topic-specific chapter, of roughly 15 pages
%
% Evaluate what you did. Highly topic-specific, but for many projects will have flavours of the following:
%     * Functional testing, including analysis and explanation of failure cases.
%     * Behavioural testing, often including analysis of any results that draw some form of conclusion wrt. the aims and objectives.
%     * Evaluation of options and decisions within the project, and/or a comparison with alternatives.
%
% This chapter often differentiates project quality. Even if the work completed is of a high technical quality, critical yet objective evaluation and comparison of the outcomes is crucial.
%
% The reader wants to learn something, so the worst examples amount to simple statements of fact e.g.'graph X shows the result is Y'. The best examples are analytical and exploratory e.g.'graph X shows the result is Y, which means Z; this contradicts Z1, which may be because I use a different assumption'. As such, both positive and negative outcomes are valid if presented in a suitable manner.

Exactly what can my algorithm do?
Show it working with different distortions.

Can it do more than described in this paper?

Where does it eventually fall down?
Show where it stops working.

What assumptions have I made? i.e. data is sync'd at t=0.

Can it do more or less, better or worse than DTW?

Parameters and what they affect:
\begin{itemize}
    \item window size
    \item step size
    \item length of signal
    \item noise of signal
    \item periodicity of signal
\end{itemize}

\section{SPHERE Data Tests}
Evaluation of SPHERE tests here.

\section{Results}
Results here.
