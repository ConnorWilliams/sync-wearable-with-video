\chapter{Technical Background}
\label{chap:technical}
% A compulsory chapter of roughly 10 pages
%
% Describe the technical basis on which execution of the project depends.
% Provide a detailed explanation of the specific problem at hand
% Existing work that is relevant (e.g.an existing algorithm that you use,
% alternative solutions proposed, supporting technologies).
%
% After reading this chapter, a non-expert reader should have obtained enough
% background knowledge to understand what you have done, then accurately assess
% your work.
%
% Give the reader confidence that you are able to absorb, understand and clearly
% communicate highly technical material.

\section{Initial solution ideas}
In a centralized system the solution is trivial; the centralized server will dictate the system time. Cristian's algorithm and the Berkeley Algorithm are some solutions to the clock synchronization problem in a centralized server environment. In a distributed system the problem takes on more complexity because a global time is not easily known. The most used clock synchronization solution on the Internet is the Network Time Protocol (NTP) which is a layered client-server architecture based on UDP message passing. Lamport timestamps and vector clocks are concepts of the logical clocks in distributed systems.

Cristian's algorithm
Cristian's algorithm relies on the existence of a time server. The time
server maintains its clock by using a radio clock or other accurate time source,
then all other computers in the system stay synchronized with it. A time client
will maintain its clock by making a procedure call to the time server. Variations
of this algorithm make more precise time calculations by factoring in network radio
propagation time.

Berkeley algorithm
The Berkeley algorithm is suitable for systems where a radio clock is not present,
this system has no way of making sure of the actual time other than by maintaining
a global average time as the global time. A time server will periodically fetch
the time from all the time clients, average the results, and then report back to
the clients the adjustment that needs be made to their local clocks to achieve
the average. This algorithm highlights the fact that internal clocks may vary
not only in the time they contain but also in the clock rate.

Often, any client whose clock differs by a value outside of a given tolerance is
disregarded when averaging the results. This prevents the overall system time
from being drastically skewed due to one erroneous clock.

Network Time Protocol
Network Time Protocol a class of mutual network synchronization protocol that
allows for use-selectable policy control in the design of the time synchronization
and evidence model. NTP supports single inline and meshed operating models in
which a clearly defined master source of time is used ones in which no penultimate
master or reference clocks are needed.

In NTP service topologies based on peering, all clocks equally participate in
the synchronization of the network by exchanging their timestamps using regular
beacon packets. In addition NTP supports a unicast type time transfer which
provides a higher level of security. NTP performance is tunable based on its
application and environmental loading as well. NTP combines a number of
algorithms to robustly select and compare clocks, together with a combination of
linear and decision-based control loop feedback models that allows multiple time
synchronization probes to be combined over long time periods to produce high
quality timing and clock drift estimates. Because NTP allows arbitrary
synchronization mesh topologies, and can withstand (up to a point) both the loss
of connectivity to other nodes, and \"falsetickers\" that do not give consistent
time, it is also robust against failure and misconfiguration of other nodes in
the synchronization mesh.

NTP is highly robust, widely deployed throughout the Internet, and well tested
over the years, and is generally regarded as the state of the art in distributed
time synchronization protocols for unreliable networks. It can reduce synchronization
offsets to times of the order of a few milliseconds over the public Internet,
and to sub-millisecond levels over local area networks.

A simplified version of the NTP protocol, SNTP, can also be used as a pure
single-shot stateless master-slave synchronization protocol, but lacks the
sophisticated features of NTP, and thus has much lower performance and reliability
levels.

Clock Sampling Mutual Network Synchronization
CS-MNS is suitable for distributed and mobile applications. It has been shown to
be scalable over mesh networks that include indirectly linked non-adjacent nodes,
and compatible to IEEE 802.11 and similar standards. It can be accurate to the order
of few microseconds, but requires direct physical wireless connectivity with
negligible link delay (less than 1 microsecond) on links between adjacent nodes,
limiting the distance between neighboring nodes to a few hundred meters.

Precision Time Protocol
Precision Time Protocol (PTP) is a master/slave protocol for delivery of highly
accurate time over local area networks

Synchronous Ethernet
Synchronous Ethernet uses Ethernet in a synchronous manner such that when
combined with synchronization protocols such as Precision Time Protocol in the
case of the White Rabbit Project, sub-nanosecond synchronization accuracy may be
achieved.

Reference broadcast synchronization
The Reference Broadcast Synchronization (RBS) algorithm is often used in
wireless networks and sensor networks. In this scheme, an initiator broadcasts a
reference message to urge the receivers to adjust their clocks.

Reference Broadcast Infrastructure Synchronization
The Reference Broadcast Infrastructure Synchronization (RBIS) protocol is a
master/slave synchronization protocol based on the receiver/receiver
synchronization paradigm, as RBS. It is specifically tailored to be used in IEEE
802.11 Wi-Fi networks configured in infrastructure mode (i.e., coordinated by an
access point). The protocol does not require any modification to the access point.

Global Positioning System
The Global Positioning System can also be used for clock synchronization. The
accuracy of GPS time signals is ±10 ns[7] and is second only to the atomic
clocks upon which they are based.
