\chapter{Contextual Background}
\label{chap:context}
% A compulsory chapter, of roughly 5 pages
%
% Describe the project context, and motivate each aim and objective.
% Fairly high-level, and easily understood by a reader who is technically
% competent but not an expert in the topic itself.
%
% Answer three questions for the reader:
%     * What is the project topic, or problem being investigated?
%     * Why is the topic important, or why should the reader care about it?
%         For example, why there is a need for this project (e.g., lack of similar
%         software or deficiency in existing software), who will benefit from the
%         project and in what way (e.g., end-users, or software developers) what
%         work does the project build on and why is the selected approach either
%         important and/or interesting (e.g., fills a gap in literature, applies
%         results from another field to a new problem).
%     * What are the central challenges involved and why are they significant?
%
% This chapter should conclude with a concise bullet point list that
% summarises the aims and objectives.  For example:
%
%     The high-level objective of this project is to reduce the performance
%     gap between hardware and software implementations of modular arithmetic.
%     More specifically, the concrete aims are:
%
%     * Research and survey literature on public-key cryptography and
%           identify the state of the art in exponentiation algorithms.
%     * Improve the state of the art algorithm so that it can be used
%           in an effective and flexible way on constrained devices.
%     * Implement a framework for describing exponentiation algorithms
%           and populate it with suitable examples from the literature on
%           an ARM7 platform.
%     * Use the framework to perform a study of algorithm performance
%           in terms of time and space, and show the proposed improvements
%           are worthwhile.

This project could be broken down in the following steps:
\begin{enumerate}
    \item Consider different types of temporal distortions which could take place
        on a device such as this one.
    \item Implement a program which generates data and can apply any combination
        of the different types of temporal distortion.
    \item Devise a method to compensate for the different temporal distortions.
    \item Given a synchronised set of data from the accelerometer and the
        cameras, introduce artificial drifts between data sets and assess how well
        the methods work.
    \item Apply this method to real drifted data which has already been
        collected in the SPHERE house in three locations; hallway, living room and kitchen.
\end{enumerate}

Implicit to all of these steps will is a fair amount of data formatting and
visualisation. I use Python for the implementation with SciPy which
is an open source library of scientific tools for Python.
