% \documentclass[options]{class}
% CLASSES:
% article     for articles in scientific journals, presentations, short reports, program documentation, invitations.
% proc        for proceedings, based on the article class.
% minimal     only sets a page size and a base font. It is mainly used for debugging purposes.
% report      for longer reports containing several chapters, small books, PhD theses.
% book        for real books
% slides      for slides. The class uses big sans serif letters. You might want to consider using the Beamer class instead.

\documentclass{article}
% Preamble
% \usepackage[options]{package}
\usepackage{graphicx}
\graphicspath{ {../} }
\usepackage{amsmath}
\usepackage[margin=1in]{geometry}
\usepackage{wrapfig}
\setlength{\parindent}{0pt}
\setlength{\parskip}{1ex plus 0.5ex minus 0.2ex}


\begin{document}
\title{Thesis Structure}
\author{Connor Williams - cw13121}
\date{}
\maketitle

Dan Page kindly provided a LaTeX template for writing a thesis. The following structure is based on Dan's template with my own bits added in.

None of the chapters are currently fully complete but I have made a good start on Contextual Background, Technical Background and Project Execution. Chapters and sections marked with (TODO) have not been started yet.

I need to do some 'torture testing' on my algorithm and describe under what conditions it starts to fall down. I also still need to test on the SPHERE data which I will do this weekend.

\section{Front Matter}
\subsection{Abstract}
\subsection{Supporting Technologies}
\subsection{Notation and Acronyms}
\subsection{Acknowledgements}

\section{Main Matter}
\subsection{Contextual Background}
\begin{itemize}
\item A high-level description of the project context to motivate each aim and objective.
\item Introduction to SPHERE: what it is, what the long term aims are, how the technology could help.
\item What the problem is: "Within the SPHERE group there is an accelerometer..."
\item Short description of the accelerometer.
\item Short description of the RGB-D cameras.
\item What the current solution is.
\item High level objective of the project.
\item A brief breakdown of the project (4 or 5 bullet points).
\item Benefits if the project is successful.
\end{itemize}
  


\subsection{Technical Background}
\begin{itemize}
\item Proposed solution 1 - Synchronize two device clocks using algorithms such as Cristian's algorithm, Berkeley algorithm or NTP. Explain why these are no suitable (i.e. no network connectivity)
\item Proposed solution 2 - Synchronise the signals along the time axis.
    \begin{itemize}
    \item Cite some papers on how people have done this before.
    \item Show I can derive acceleration from position: f(x) = position, f'(x) = velocity, f''(x) = acceleration.
    \item How do you track a drift?
    \item What does 'drift'/ 'temporal distortion' even mean?
    \item Define types of temporal distortion we will be working with.
    \item Introduce cross correlation and explain why it wont work 'out of the box'.
    \item Introduce the idea of a sliding window cross correlation.
    \end{itemize}    
\end{itemize}


\subsection{Project Execution}
\begin{itemize}
\item Highlight best practice such as the fact I used version control, this blog.
\item Talk about why I chose Python and SciPy.
\item Description in chronological order of how the project started out: obtained clean SPHERE data, visualised it, did some processing, decided it was too noisy and complex.
\item Decided to generate my own data.
\item Describe my data generator, the features it has and why, different parameters users can change.
\item Explain how this data made it easier to solve the task at hand.
\item Describe the process from this synthetic data and the plots I used to eventually be successful in correcting for the different time drifts.
\item Pseudo-code for the algorithm.
\item Show the algorithm working for different time drifts on synthetic data.
\item Show how the algorithm performs when I add temporal distortions to the clean SPHERE data (TO DO).
\end{itemize}


\subsection{Critical Evaluation (TO DO)}
\begin{itemize}
\item Describe exactly what can my algorithm do? (Show extreme cases of it working)
\item Explain if it can it do more than/ less than/ exactly what we originally aimed for?
\item Highlight the limitations of the algorithm and describe the conditions where it fails.
\item Talk about assumptions I have made. (e.g. data is synchronised at t=0)
\item Describe the parameters that a user can change and what they affect: window size, step size, length of signal to use.
\item Evaluate how the algorithm works on the SPHERE data.
\item Describe how it could be made better.
\end{itemize}
  


\subsection{Conclusion and Further Work (TO DO)}
\begin{itemize}
\item (Re)summarise the main contributions and achievements, summing up the content.
\item Clearly state the current project status e.g. 'I have an algorithm which can correct for these types of drift which could help/ has helped us discover X about the SPHERE data'.
\item Evaluate what has been achieved with respect to the initial aims and objectives e.g. 'I completed aim X outlined previously, the evidence for this is within
    Chapter Y'. Also include that I did not get as far as initially planned in fixing the accelerometer problem due to finding so many complications.
\item Outline any open problems or future plans such as this algorithm could be used in SPHERE to reject false positive skeletons found by the RGB-D cameras etc.
\item What I could have done given more time.
\end{itemize}
  


\end{document}
